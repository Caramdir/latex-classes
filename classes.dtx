% \iffalse meta-comment
%<*internal>
\def\nameofplainTeX{plain}
\ifx\fmtname\nameofplainTeX\else
  \expandafter\begingroup
\fi
%</internal>
%<*install>
\input docstrip.tex
\keepsilent
\askforoverwritefalse
\preamble
Copyright (C) 2011 by Clemens Koppensteiner
-------------------------------------------

This file may be distributed and/or modified under the
conditions of the LaTeX Project Public License, either version 1.3
of this license or (at
 your option) any later version.
The latest version of this license is in:

   http://www.latex-project.org/lppl.txt

and version 1.3 or later is part of all distributions of LaTeX
version 2005/12/01 or later.
\endpreamble
\generate{\file{short-notes.cls} {\from{classes.dtx}{short-notes,common}}}
\generate{\file{lecture-notes.cls} {\from{classes.dtx}{lecture-notes,common}}}
\generate{\file{paper-notes.cls} {\from{classes.dtx}{paper-notes,common}}}
%</install>
%<install>\endbatchfile
%<*internal>
\generate{
  \file{\jobname.ins}{\from{\jobname.dtx}{install}}
}
\ifx\fmtname\nameofplainTeX
  \expandafter\endbatchfile
\else
  \expandafter\endgroup
\fi
%</internal>
%
%<common>\NeedsTeXFormat{LaTeX2e}[1999/12/01]
%<short-notes>\ProvidesClass{short-notes}
%<lecture-notes>\ProvidesClass{lecture-notes}
%<paper-notes>\ProvidesClass{paper-notes}
%<common>   [2011/07/22 v0.2
%<short-notes>     A class for short notes
%<lecture-notes>   A class for lecture notes
%<paper-notes>     A class for notes on articles
%<common>   ]
%
%<*driver>
\documentclass{ltxdoc}
\usepackage[T1]{fontenc}
\usepackage{csquotes,dtklogos}
\usepackage{lmodern}
\usepackage[numbered]{hypdoc}
\EnableCrossrefs
\CodelineIndex
\RecordChanges
\begin{document}
  \DocInput{\jobname.dtx}
\end{document}
%</driver>
% \fi
%
% \changes{v0.2}{2011/07/22}{Converted to DTX.}
% 
% \PrintChanges
%
% \section{Load a base class}
%
%    \begin{macrocode}
%<lecture-notes>\LoadClassWithOptions{scrbook}
%<!lecture-notes>\LoadClassWithOptions{scrartcl}
%    \end{macrocode}
%
%<*common>
%
% Load some packages that are needed to make decisions
%    \begin{macrocode}
\RequirePackage{xstring}
\RequirePackage{iftex}
%    \end{macrocode}
%
% \section{Process options and set flags}
%
%    \begin{macrocode}
\newif\ifkindledxlandscape
\newif\ifkindledxportrait
\newif\ifkindledx
\newif\ifbw
\newif\iffinal
%    \end{macrocode}
%
% Try to set some options based on the name of the input file.
%    \begin{macrocode}
\IfEndWith{\jobname}{\detokenize{-kindledxlandscape}}{\kindledxlandscapetrue}{}
\IfEndWith{\jobname}{\detokenize{-kindledxportrait}}{\kindledxportraittrue}{}
\IfEndWith{\jobname}{\detokenize{-bw}}{\bwtrue}{}
%    \end{macrocode}
%
% Set options based on the options passed to the class.
%    \begin{macrocode}
\DeclareOption{kindledxlandscape}{
    \kindledxlandscapetrue
}
\DeclareOption{kindledxportrait}{
    \kindledxportraittrue
}
\DeclareOption{bw}{
    \bwtrue
}    
\DeclareOption{final}{
    \finaltrue
}

\ProcessOptions\relax

\ifkindledxportrait\kindledxtrue\fi
\ifkindledxlandscape\kindledxtrue\fi
\ifkindledx\bwtrue\fi
%    \end{macrocode}
%
% \section{Set up the page geometry}
%
% \texttt{footinclude} determines whether the footer is considered as part of the text for the layout calculation.
% \texttt{headinclude} is the same for the header.
% \texttt{pagesize} tells KOMA to correctly specify the pdf page size.
%    \begin{macrocode}
\KOMAoptions{
    fontsize=10pt,
    footinclude=false, 
    headinclude=false,
    pagesize,
%    \end{macrocode}
%</common>
% \\ For the lecture notes class we want chapter to open on the right page only.\\
%<*lecture-notes>
%    \begin{macrocode}
    twoside,
    open=right,
%    \end{macrocode}
%</lecture-notes>
%<*common>
%    \begin{macrocode}
}
\recalctypearea
%    \end{macrocode}
% 
% Set the size of the Kindle DX versions manually.
%    \begin{macrocode}
\ifkindledxlandscape
    \RequirePackage[papersize={19.5cm,12.5cm},lmargin=0.2cm,rmargin=5cm,marginparwidth=4.5cm,marginparsep=0.3cm,bottom=0.2cm,top=0.8cm, headsep=0.3cm,footskip=0cm,twoside=false]{geometry}
\fi

\ifkindledxportrait
    \RequirePackage[papersize={13.3cm,18.8cm},lmargin=0.2cm,rmargin=0.5cm,bottom=0.2cm,top=0.8cm,headsep=0.3cm,footskip=0cm,twoside=false]{geometry}
\fi
%    \end{macrocode}
%
% \section{Load various packages}
%
% The KOMA package for setting up the page header and footer.
%    \begin{macrocode}
\RequirePackage{scrpage2}
%    \end{macrocode}
%
% The \AMS-math bundle and \texttt{thmtools} for specifying theorem environments.
%    \begin{macrocode}
\RequirePackage{amsmath,amsfonts,amssymb,amsthm,thmtools}
%    \end{macrocode}
% Various math-related packages to extend functionality (e.g. \cs{mathclap}, extensible arrows and slanted fractions).
%    \begin{macrocode}
\RequirePackage{mathtools,extpfeil,xfrac}
%    \end{macrocode}
%
% \texttt{mparhack} tries to place margin notes on the correct side of the page when they are moved across a page boundary.
%    \begin{macrocode}
\RequirePackage{mparhack}  % correct placement of margin notes
%    \end{macrocode}
% Don't show ToDo notes in the final version of a document.
%    \begin{macrocode}
\iffinal
    \RequirePackage[disable]{todonotes}
\else
    \RequirePackage{todonotes}
\fi
%    \end{macrocode}
% Macro for properly nesting quotes.
%    \begin{macrocode}
\RequirePackage{csquotes}
%    \end{macrocode}
%
% Load \texttt{babel} or \texttt{polyglossia}, depending on the engine.
% In the \texttt{babel} case, the language should be passed via the class options.
%    \begin{macrocode}
\ifXeTeX
    \RequirePackage{polyglossia}
    \setmainlanguage{english}
\else
    \RequirePackage{babel}
\fi
%    \end{macrocode}
% List environments.
%    \begin{macrocode}
\RequirePackage{paralist}
%    \end{macrocode}
% English ordinal numbers
%    \begin{macrocode}
\RequirePackage{engord}
%    \end{macrocode}
% Telling \LaTeX\ that the next code needs a certain amount of space to avoid page breaks at bad places.
%    \begin{macrocode}
\RequirePackage{needspace}
%    \end{macrocode}
% Various \TeX\ logos.
%    \begin{macrocode}
\RequirePackage{dtklogos}
%    \end{macrocode}
%
% Bibliographies
%    \begin{macrocode}
\RequirePackage[backend=biber]{biblatex}
%    \end{macrocode}
%
% Wrappers around \eTeX\ primitives.
%    \begin{macrocode}
\RequirePackage{etoolbox}
%    \end{macrocode}
%
% If using \LuaTeX, load some fixes for math.
%    \begin{macrocode}
\ifluatex
    \RequirePackage{lualatex-math}
\fi
%    \end{macrocode}
%
% \section{Fonts}
%
% We use \texttt{fontspec} to load the text fonts and \texttt{unicode-math} for math.
% The various \cs{ck@...font} macros hold the font names so that they can be referred to in the text.
%    \begin{macrocode}
\RequirePackage{unicode-math}

\def\ck@mainfont{TeX Gyre Termes}
\setmainfont[
  Ligatures=TeX,
  Extension=.otf,
  UprightFont= *-regular,
  BoldFont=*-bold,
  ItalicFont=*-italic,
  BoldItalicFont=*-bolditalic,
  Scale=MatchUppercase
]{texgyretermes}

\def\ck@sansfont{TeX Gyre Adventor}
\setsansfont[
  Ligatures=TeX,
  Extension=.otf,
  UprightFont= *-regular,
  BoldFont=*-bold,
  ItalicFont=*-italic,
  BoldItalicFont=*-bolditalic,
  Scale=MatchUppercase
]{texgyreadventor}

\def\ck@monofont{Liberation Mono}
\setmonofont[
  Ligatures=TeX,
  Scale=MatchLowercase
]{Liberation Mono}

\def\ck@mathfont{XITS Math}
\setmathfont{xits-math.otf}
%    \end{macrocode}
% 
% \section{Sectioning}
%
%    \begin{macrocode}
\ifbw
    \colorlet{sectioningcolor}{black}
\else
    \colorlet{sectioningcolor}{cyan!15!blue}
\fi
%    \end{macrocode}
%
%</common>
%
% \subsection{Lecture notes}
%<*lecture-notes>
%    \begin{macrocode}
\ifbw
    \setkomafont{chapter}{\normalfont\normalcolor\sffamily\scshape\huge\color{sectioningcolor}}
    \setkomafont{section}{\normalfont\normalcolor\sffamily\scshape\Large\color{sectioningcolor}}
    \setkomafont{subsection}{\normalfont\normalcolor\sffamily\scshape\large\color{sectioningcolor}}
\else
    \setkomafont{chapter}{\normalfont\normalcolor\sffamily\huge\color{sectioningcolor}}
    \setkomafont{section}{\normalfont\normalcolor\sffamily\Large\color{sectioningcolor}}
    \setkomafont{subsection}{\normalfont\normalcolor\sffamily\large\color{sectioningcolor}}
\fi
%    \end{macrocode}
%</lecture-notes>
%
% \subsection{Other classes}
%<*!lecture-notes>
%    \begin{macrocode}
\KOMAoptions{numbers=enddot}

\ifbw
    \setkomafont{section}{\normalfont\normalcolor\sffamily\scshape\Large\color{sectioningcolor}}
    \setkomafont{subsection}{\normalfont\normalcolor\sffamily\scshape\large\color{sectioningcolor}}
\else
    \setkomafont{title}{\normalfont\normalcolor\scshape\LARGE\color{sectioningcolor}}
    \setkomafont{section}{\normalfont\normalcolor\scshape\large\color{sectioningcolor}\centering}
    \setkomafont{subsection}{\normalfont\normalcolor\scshape\color{sectioningcolor}}
\fi
%    \end{macrocode}
%</!lecture-notes>
%
% \subsection{Common options}
%<*common>
%    \begin{macrocode}
\setkomafont{paragraph}{\normalfont\normalcolor\sffamily\bfseries}
\setkomafont{subparagraph}{\normalfont\normalcolor\sffamily\itshape}
%    \end{macrocode}
% 
% \section{Header and footer}
%
%    \begin{macrocode}
\pagestyle{scrheadings}
%    \end{macrocode}
%
%</common>
% \subsection{Lecture notes}
%<*lecture-notes>
%    \begin{macrocode}

\ifkindledx
    \AtBeginDocument{%
        \cfoot{}%
        \chead{\normalfont\scshape\headmark}%
        \ohead{\normalfont\pagemark}%
        \renewcommand*\partpagestyle{empty}%
        \renewcommand*\chapterpagestyle{empty}%
        \renewcommand*\indexpagestyle{empty}%
    }
\else
    \AtBeginDocument{%
        \lehead{\normalfont\scshape\headmark}%
        \setheadsepline{0.4pt}[\color{black!30}]%
    }
    \appto\backmatter{%
		\rohead{\normalfont\scshape\headmark}%
    }
\fi
%    \end{macrocode}
%</lecture-notes>
%
% \subsection{Other classes}
%<*!lecture-notes>
%    \begin{macrocode}
\AtBeginDocument{
    \clearscrheadfoot
    \let\@@title\@title
    \ihead{\normalfont\scshape\color{gray}\@@title}
    \chead{}
    \ohead{\pagemark}
    \cfoot[\pagemark]{}
    \setheadsepline{0.5pt}
}
%    \end{macrocode}
%</!lecture-notes>
%
%<*common>
%
% \section{Index and List of Notation}
% 
% These are only needed in the lecture notes class.\\
%</common>
%<*lecture-notes>
%    \begin{macrocode}
\RequirePackage{makeidx}
\RequirePackage{index}
\makeindex
\appto\theindex{%
	\markboth{\indexname}{\indexname}%
	\addcontentsline{toc}{chapter}{\indexname}%
}

\RequirePackage[noprefix,refpage,intoc]{nomencl}
\renewcommand\nomname{List of Notation}
\renewcommand{\pagedeclaration}[1]{\hfill #1}
\makenomenclature
%    \end{macrocode}
%</lecture-notes>
%<*common>
%
% \section{Various things}
%
% \subsection{TikZ}
%    \begin{macrocode}
\RequirePackage{tikz}
\usetikzlibrary{
    positioning,
    calc,
    arrows,
    decorations.markings,
    decorations.pathmorphing,
    decorations.pathreplacing,
    matrix,
    intersections}
\tikzstyle{every picture}+=[>=latex]
\tikzset{commutative diagram/.style={
    matrix of math nodes,
    column sep=2em,
    row sep=2em,
    text height=1.5ex,
    text depth=0.25ex}}
\tikzset{spaced commutative diagram/.style={
    commutative diagram,    
    column sep=3em,
    row sep=3em}}
\tikzset{commutative diagram arrows/.style={
    ->,
    font=\small}}
\tikzset{descr/.style={fill=white,inner sep=1pt}}
%    \end{macrocode}
% 
% \subsection{Workaround for the \texorpdfstring{\XeTeX}{XeTeX} underbrace bug}
%
% See \href{https://github.com/wspr/unicode-math/issues/169}{the bug report}.
%    \begin{macrocode}
\ifxetex
    \def\underbrace#1{\@ifnextchar_{\tikz@@underbrace{#1}}{\tikz@@underbrace{#1}_{}}}
    \def\tikz@@underbrace#1_#2{%
        \tikz[baseline=(a.base)] {
            \node[inner xsep=0,inner ysep=1pt] (a) {\(#1\)};
            \draw[thick,decorate,decoration={brace,amplitude=5pt}] (a.south east) -- 
                node[below,inner sep=7pt] {\(\scriptstyle #2\)} (a.south west);
        }%
    }
\fi
%    \end{macrocode}
%
% \subsection{Margin notes}
%    \begin{macrocode}
\ifkindledxportrait
    \newcommand\remark{\footnote}
\else
    \newcommand\remark[1]{\leavevmode\marginpar{\footnotesize\itshape #1}}
\fi

\newcommand\sidepicture[2][]{\marginpar{#2\\[0.5em]{\footnotesize\itshape #1}}}
%    \end{macrocode}
%
% \subsection{Remap some characters}
% Remap non-breaking spaces to normal space (I sometimes type them accidentally on the Neo keyboard layout).
%    \begin{macrocode}
\RequirePackage{newunicodechar}
\newunicodechar{ }{ } % no-break space
\newunicodechar{ }{ } % narrow no-break space
%    \end{macrocode}
%
% \section{Load \texttt{hyperref}}
%
%    \begin{macrocode}
\RequirePackage[colorlinks=false,unicode=true]{hyperref}
%    \end{macrocode}
% 
% \section{Set up theorem-like environments}
%
% For the lecture notes class, the theorem counter should be reset every chapter.
% For the shorter notes, it shouldn't be reset.
%
%    \begin{macrocode}
\theoremstyle{plain}
%    \end{macrocode}
%</common>
%    \begin{macrocode}
%<lecture-notes>\newtheorem{Thm}{Theorem}[chapter]
%<!lecture-notes>\newtheorem{Thm}{Theorem}
%    \end{macrocode}
%<*common>
%    \begin{macrocode}
\newtheorem{Fact}[Thm]{Fact}
\newtheorem{Prop}[Thm]{Proposition}
\newtheorem{Cor}[Thm]{Corollary}
\newtheorem{CorAd}[Thm]{Corollary*}
\newtheorem{Lem}[Thm]{Lemma}
\newtheorem{LemAd}[Thm]{Lemma*}
\newtheorem{Alg}[Thm]{Algorithm}
\newtheorem{DefProp}[Thm]{Definition and Proposition}

\theoremstyle{definition}
\newtheorem{Def}[Thm]{Definition}
\newtheorem*{Notation}{Notation}

\theoremstyle{remark}
\newtheorem{Rem}[Thm]{Remark}
\newtheorem{RemAd}[Thm]{Remark*}
\newtheorem{Rems}[Thm]{Remarks}
\newtheorem{Exercise}[Thm]{Exercise}

\declaretheorem[name=Example,sibling=Thm,qed={\lower-0.3ex\hbox{$◃$}}]{Ex}
\declaretheorem[name=Examples,sibling=Ex,qed={\lower-0.3ex\hbox{$◃$}}]{Exs}
%    \end{macrocode}
%
% \section{The title page}
%</common>
% \subsection{Lecture notes}
%<*lecture-notes>
%
% Some commands to set and store various metadata
%    \begin{macrocode}
\let\@author\@empty%
\let\@course\@empty%
\newcommand{\course}[1]{\gdef\@course{#1}}%
\let\@courseinfo\@empty%
\newcommand{\courseinfo}[1]{\gdef\@courseinfo{#1}}%
\let\@typedby\@empty%
\newcommand{\typedby}[1]{\gdef\@typedby{#1}}%
\let\@version\@empty%
\newcommand{\version}[1]{\gdef\@version{#1}}%
\let\@disclaimer\@empty%
\newcommand{\disclaimer}[1]{\gdef\@disclaimer{#1}}%
%    \end{macrocode}
%
% Store some of that data in the pdf.
%    \begin{macrocode}
\AtBeginDocument{%
	\hypersetup{
		pdfinfo={
			Title={\@title},
			Author={\ifx\@author\@empty\ifx\@typedby\@empty\else\@typedby\fi\else\@author\fi}
		}
	}%
}
%    \end{macrocode}
% 
% The actual title page.
%    \begin{macrocode}
\newcommand\maketitlepage{%
\begin{titlepage}
    \sffamily
    \ifkindledx\relax\else\vspace*{3cm}\fi

    \noindent
    \huge\color{sectioningcolor}
    \@title\\[0.6cm]
    \ifx\@course\@empty \else
      \normalcolor\large
      based on the course\\
      \emph{\@course}%
      \ifx\@courseinfo\@empty\else\\\@courseinfo\fi\\[1cm]
    \fi
    \Large
    \ifx\@typedby\@empty{\ifx\@author\@empty\else\@author\\[1cm]\fi}\else
      written by \@typedby\\[1cm]
    \fi
    \normalcolor
    \large
    \ifx\@version\@empty\else
      Version \@version\\
    \fi
    \ifx\@date\today\else\@date\fi

\clearpage\normalfont\normalsize\normalcolor
\thispagestyle{plain}
\noindent
\ifx\@disclaimer\@empty\else\@disclaimer\\[1cm]\fi
The current version of this text as well as the \LaTeX{} source code can be found
at \url{http://caramdir.at/math}.

\ifkindledxlandscape%
\noindent This version of the document has been specially formatted to fit on a Kindle DX (second generation) in landscape rotation. 
For optimal viewing select \enquote{actual size} from the text menu.
\fi
\ifkindledxportrait%
\noindent This version of the document has been specially formatted to fit on a Kindle DX (second generation).
\fi
\vfill

\noindent
This document was typeset with
\ifluatex Lua\LaTeX\fi\ifxetex \XeLaTeX\fi{} and KOMA-Script.
The used fonts are \ck@mainfont, \ck@sansfont, \ck@monofont{} and \ck@mathfont.
\vspace{1cm}

\noindent
This work is licensed under the Creative Commons Attribution-Share Alike 3.0 Austria License.
To view a copy of this license, visit \url{http://creativecommons.org/licenses/by-sa/3.0/at/}
or send a letter to\\
Creative Commons\\
171 Second Street, Suite 300\\
San Francisco\\
California, 94105\\
USA.
\end{titlepage}
\cleardoublepage
}
%    \end{macrocode}
%</lecture-notes>
%
% \subsection{Other classes}
%
% For \texttt{paper-notes.sty} create the title from a citation.\\
%<*paper-notes>
%    \begin{macrocode}
\let\ck@paperref\@empty
\def\paperref#1{%
    \def\ck@paperref{#1}%
    \title{Notes on: \citeauthor{#1} --- \citetitle{#1}}%
}
%    \end{macrocode}
%</paper-notes>
%
%<*!lecture-notes>
%    \begin{macrocode}
\AtBeginDocument{%
	\hypersetup{
		pdfinfo={
			Title={\@title},
			Author={\@author}
		}
	}%
}

\renewcommand*{\@maketitle}{%
  \clearpage
  \let\footnote\thanks
  \ifx\@extratitle\@empty \else
    \noindent\@extratitle \next@tpage \if@twoside \null\next@tpage \fi
  \fi
  \setparsizes{\z@}{\z@}{\z@\@plus 1fil}\par@updaterelative
  \ifx\@titlehead\@empty \else
    \begin{minipage}[t]{\textwidth}
      \@titlehead
    \end{minipage}\par
  \fi
  \null
  \vskip 2em%
  \begin{center}%
    \ifx\@subject\@empty \else
      {\subject@font \@subject \par}
      \vskip 1.5em
    \fi
%    \end{macrocode}
%<*paper-notes>
%    \begin{macrocode}
    {\titlefont%
     {\normalcolor Notes on:}\\%
     {\Large\citeauthor{\ck@paperref}\\}%
     \citetitle{\ck@paperref}%
     {\normalcolor\cite{\ck@paperref}}\par%
    }%
%    \end{macrocode}
%</paper-notes>
%<*short-notes>
%    \begin{macrocode}
    {\titlefont \@title \par}%
%    \end{macrocode}
%</short-notes>
%    \begin{macrocode}
    \vskip .5em
    {\ifx\@subtitle\@empty\else\usekomafont{subtitle}\@subtitle\par\fi}%
    \vskip 1em
    {\scshape
      \lineskip .5em%
      \begin{tabular}[t]{c}
        \@author
      \end{tabular}\par
    }%
    {\scshape \@date \par}%
    \vskip \z@ \@plus 1em
    {\Large \@publishers \par}
    \ifx\@dedication\@empty \else
      \vskip 2em
      {\Large \@dedication \par}
    \fi
  \end{center}%
  \par
  \vskip 1em
}%
%    \end{macrocode}
%</!lecture-notes>
%<*common>
%
%</common>
% \Finale
